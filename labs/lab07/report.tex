mainfont: DejaVu Serif sansfont: DejaVu Sans monofont: DejaVu Sans Mono
lang: ru --- title: ``лабораторная работа №7'' author: ``Сархан хабиба
осама'' date: ``22 ноября 2025'' ---

\hypertarget{ux446ux435ux43bux44c-ux440ux430ux431ux43eux442ux44b}{%
\section{Цель
работы}\label{ux446ux435ux43bux44c-ux440ux430ux431ux43eux442ux44b}}

Освоение арифметических инструкций языка ассемблера NASM. Изучение
команд условного и безусловного переходов. Приобретение навыков
написания программ с использованием переходов. Знакомство с назначением
и структурой файла листинга.

\hypertarget{ux445ux43eux434-ux440ux430ux431ux43eux442ux44b}{%
\section{Ход
работы}\label{ux445ux43eux434-ux440ux430ux431ux43eux442ux44b}}

\hypertarget{ux441ux43eux437ux434ux430ux43dux438ux435-ux43aux430ux442ux430ux43bux43eux433ux430-ux434ux43bux44f-ux43bux430ux431ux43eux440ux430ux442ux43eux440ux43dux43eux439-ux440ux430ux431ux43eux442ux44b}{%
\subsection{1. Создание каталога для лабораторной
работы}\label{ux441ux43eux437ux434ux430ux43dux438ux435-ux43aux430ux442ux430ux43bux43eux433ux430-ux434ux43bux44f-ux43bux430ux431ux43eux440ux430ux442ux43eux440ux43dux43eux439-ux440ux430ux431ux43eux442ux44b}}

\begin{Shaded}
\begin{Highlighting}[]
\FunctionTok{mkdir}\NormalTok{ \textasciitilde{}/work/arch{-}pc/lab07}
\BuiltInTok{cd}\NormalTok{ \textasciitilde{}/work/arch{-}pc/lab07}
\FunctionTok{touch}\NormalTok{ lab7{-}1.asm}
\end{Highlighting}
\end{Shaded}

\begin{figure}
\centering
\includegraphics{./1.png}
\caption{Результат lab7}
\end{figure}

\begin{enumerate}
\def\labelenumi{\arabic{enumi}.}
\setcounter{enumi}{1}
\tightlist
\item
  Создание и выполнение программы lab7-1.asm Создан файл lab7-1.asm с
  программой безусловных переходов:
\end{enumerate}

\begin{verbatim}
%include 'in_out.asm'

SECTION .data
msg1: DB 'Сообщение № 1',0
msg2: DB 'Сообщение № 2',0
msg3: DB 'Сообщение № 3',0

SECTION .text
GLOBAL _start
_start:
    jmp _label2
_label1:
    mov eax, msg1
    call sprintLF
    jmp _end
_label2:
    mov eax, msg2
    call sprintLF
    jmp _label1
_label3:
    mov eax, msg3
    call sprintLF
_end:
    call quit
\end{verbatim}

Компиляция и выполнение:

\begin{Shaded}
\begin{Highlighting}[]

\FunctionTok{nasm} \AttributeTok{{-}f}\NormalTok{ elf lab7{-}1.asm}
\FunctionTok{ld} \AttributeTok{{-}m}\NormalTok{ elf\_i386 }\AttributeTok{{-}o}\NormalTok{ lab7{-}1 lab7{-}1.o}
\ExtensionTok{./lab7{-}1}
\end{Highlighting}
\end{Shaded}

\includegraphics{./2.png} Результат: Сообщение № 2 Сообщение № 3
;\ldots\ldots\ldots\ldots\ldots\ldots\ldots.

2\_2. Изменение программы для вывода: Сообщение № 3, Сообщение № 2,
Сообщение № 1

\begin{verbatim}
_start:
    jmp _label3
_label1:
    mov eax, msg1
    call sprintLF
    jmp _end
_label2:
    mov eax, msg2
    call sprintLF
    jmp _label1
_label3:
    mov eax, msg3
    call sprintLF
    jmp _label2
_end:
    call quit
\end{verbatim}

Результат выполнения:

Сообщение № 3 Сообщение № 2 Сообщение № 1

\includegraphics{./3.png}
;\ldots\ldots\ldots\ldots\ldots\ldots\ldots\ldots\ldots\ldots\ldots\ldots\ldots\ldots\ldots\ldots\ldots.
3. Создание и выполнение программы lab7-2.asm Код программы:

\begin{verbatim}
%include 'in_out.asm'

section .data
msg1 db 'Введите B: ',0h
msg2 db "Наибольшее число: ",0h
A dd 20
C dd 50

section .bss
max resb 10
B resb 10

section .text
global _start
_start:
    mov eax, msg1
    call sprint
    mov ecx, B
    mov edx, 10
    call sread
    mov eax, B
    call atoi
    mov [B], eax
    mov ecx, [A]
    mov [max], ecx
    cmp ecx, [C]
    jg check_B
    mov ecx, [C]
    mov [max], ecx
check_B:
    mov eax, max
    call atoi
    mov [max], eax
    mov ecx, [max]
    cmp ecx, [B]
    jg fin
    mov ecx, [B]
    mov [max], ecx
fin:
    mov eax, msg2
    call sprint
    mov eax, [max]
    call iprintLF
    call quit
\end{verbatim}

Тестирование программы:

\begin{itemize}
\tightlist
\item
  При B=30 → Результат: 50
\item
  При B=60 → Результат: 60
\item
  При B=10 → Результат: 50
\end{itemize}

\includegraphics{./4.png}
;\ldots\ldots\ldots\ldots\ldots\ldots\ldots\ldots\ldots\ldots\ldots\ldots\ldots\ldots{}

\begin{enumerate}
\def\labelenumi{\arabic{enumi}.}
\setcounter{enumi}{3}
\tightlist
\item
  Работа с файлом листинга Создание файла листинга:
\end{enumerate}

\begin{Shaded}
\begin{Highlighting}[]
\FunctionTok{nasm} \AttributeTok{{-}f}\NormalTok{ elf }\AttributeTok{{-}l}\NormalTok{ lab7{-}2.lst lab7{-}2.asm}
\end{Highlighting}
\end{Shaded}

\begin{figure}
\centering
\includegraphics{./5.png}
\caption{Результат lab7}
\end{figure}

Анализ трех строк листинга: Строка 8:

\begin{figure}
\centering
\includegraphics{./6.png}
\caption{Результат lab7}
\end{figure}

8 00000000 B800000000 mov eax, msg1

\begin{verbatim}
Адрес: 00000000

Машинный код: B800000000

Инструкция: загрузка адреса msg1 в регистр eax
\end{verbatim}

Строка 9: text

9 00000005 E8{[}00000000{]} call sprint

\begin{verbatim}
Адрес: 00000005
\end{verbatim}

Машинный код: E8{[}00000000{]}

\begin{verbatim}
Инструкция: вызов подпрограммы sprint
\end{verbatim}

Строка 10: text

10 0000000A B900000000 mov ecx, B

\begin{verbatim}
Адрес: 0000000A

Машинный код: B900000000

Инструкция: загрузка адреса B в регистр ecx
\end{verbatim}

\begin{enumerate}
\def\labelenumi{\arabic{enumi}.}
\setcounter{enumi}{6}
\item
  Анализ дополнительных строк листинга Строка 8: cmp byte {[}eax{]}, 0

  Адрес в памяти: определяется во время выполнения

  Машинный код: зависит от адреса

  Инструкция: сравнение байта по адресу в eax с нулем

  Назначение: проверка достижения конца строки (нулевой байт)
\end{enumerate}

Строка 9: jz finished

\begin{verbatim}
Адрес в памяти: следующий за cmp

Машинный код: 74 [смещение]

Инструкция: переход если равно нулю (Zero Flag = 1)

Назначение: если текущий символ = 0, переход на метку finished
\end{verbatim}

Строка 10: inc eax

\begin{verbatim}
Адрес в памяти: следующий за jz

Машинный код: 40

Инструкция: увеличение значения в eax на 1

Назначение: переход к следующему байту в строке
\end{verbatim}

Строка 11: jmp nextchar

\begin{verbatim}
Адрес в памяти: следующий за inc

Машинный код: EB [смещение]

Инструкция: безусловный переход

Назначение: возврат к началу цикла для проверки следующего символа
\end{verbatim}

Изменения в коде:

\begin{verbatim}
; Было (правильно):
mov eax, msg1

; Стало (ошибка):
mov eax     ; Удален второй операнд msg1
\end{verbatim}

Попытка компиляции с ошибкой:

\begin{verbatim}
nasm -f elf -l lab7-2-error.lst lab7-2.asm
\end{verbatim}

Результат компиляции с ошибкой:

\begin{verbatim}
lab7-2.asm:15: error: parser: instruction expected
\end{verbatim}

\includegraphics{./7.png} \includegraphics{./8.png}
\includegraphics{./9.png}

Анализ результатов при ошибке:

Какие файлы создаются:

\begin{verbatim}
 lab7-2.o - не создается

 lab7-2-error.lst - не создается

 Только сообщения об ошибках на экране
\end{verbatim}

Что добавляется в листинг:

\begin{verbatim}
При наличии ошибок файл листинга не создается вообще

Транслятор выводит ошибки только на экран
\end{verbatim}

;\ldots\ldots\ldots\ldots\ldots\ldots\ldots\ldots\ldots\ldots\ldots\ldots\ldots\ldots\ldots\ldots\ldots\ldots\ldots\ldots\ldots\ldots\ldots\ldots\ldots\ldots\ldots\ldots\ldots\ldots{}

\begin{enumerate}
\def\labelenumi{\arabic{enumi}.}
\setcounter{enumi}{3}
\tightlist
\item
  Создание и выполнение программы lab7-3.asm
\end{enumerate}

Код программы:

\begin{verbatim}
%include 'in_out.asm'

section .data
msg1 db 'Введите B: ',0h
msg2 db "Наименьшее число: ",0h
A dd 17
C dd 23

section .bss
min resb 10
B resb 10

section .text
global _start
_start:
    mov eax, msg1
    call sprint
    mov ecx, B
    mov edx, 10
    call sread
    mov eax, B
    call atoi
    mov [B], eax
    mov ecx, [A]
    mov [min], ecx
    cmp ecx, [C]
    jl check_B
    mov ecx, [C]
    mov [min], ecx
check_B:
    mov eax, min
    call atoi
    mov [min], eax
    mov ecx, [min]
    cmp ecx, [B]
    jl fin
    mov ecx, [B]
    mov [min], ecx
fin:
    mov eax, msg2
    call sprint
    mov eax, [min]
    call iprintLF
    call quit
\end{verbatim}

Тестирование программы:

\begin{itemize}
\tightlist
\item
  При B=30 → Результат: 50
\item
  При B=60 → Результат: 60
\item
  При B=10 → Результат: 50
\end{itemize}

\includegraphics{./10.png}
;\ldots\ldots\ldots\ldots\ldots\ldots\ldots\ldots\ldots\ldots\ldots\ldots\ldots\ldots\ldots\ldots\ldots.

\begin{enumerate}
\def\labelenumi{\arabic{enumi}.}
\setcounter{enumi}{4}
\tightlist
\item
  Создание и выполнение программы lab7-4.asm
\end{enumerate}

Код программы:

\begin{verbatim}
%include 'in_out.asm'

section .data
msg_x db 'Введите x: ',0h
msg_a db 'Введите a: ',0h
msg_result db "Результат f(x): ",0h

section .bss
x resb 10
a resb 10
result resb 4

section .text
global _start
_start:
    mov eax, msg_x
    call sprint
    mov ecx, x
    mov edx, 10
    call sread
    mov esi, x
    call remove_newline
    mov eax, x
    call atoi
    mov [x], eax
    mov eax, msg_a
    call sprint
    mov ecx, a
    mov edx, 10
    call sread
    mov esi, a
    call remove_newline
    mov eax, a
    call atoi
    mov [a], eax
    mov eax, [x]
    mov ebx, [a]
    cmp eax, ebx
    jl less_than
    jmp greater_equal
less_than:
    mov eax, [a]
    mov ebx, 2
    mul ebx
    sub eax, [x]
    jmp print_result
greater_equal:
    mov eax, 8
print_result:
    mov [result], ea
    mov eax, msg_result
    call sprint
    mov eax, [result]
    call iprintLF
    call quit
remove_newline:
    mov edi, esi
.next_char:
    mov al, [edi]
    inc edi
    cmp al, 0
    je .done
    cmp al, 10
    jne .next_char
    dec edi
    mov byte [edi], 0
.done:
    ret
\end{verbatim}

Тестирование программы:

\begin{itemize}
\tightlist
\item
  При x=1, a=2 → Результат: 3
\item
  При x=2, a=1 → Результат: 8
\end{itemize}

\begin{figure}
\centering
\includegraphics{./11.png}
\caption{Результат lab7}
\end{figure}

Функция:

f(x) = \{ 2a - x, если x \textless{} a \{ 8, если x ≥ a Тестирование с
значениями из таблицы:

\begin{verbatim}
При x=1, a=2 → Результат: 3 (2*2 - 1 = 3)

При x=2, a=1 → Результат: 8 (x ≥ a)
\end{verbatim}

Логика программы:

\begin{verbatim}
Ввод значений x и a

Сравнение x и a

Если x < a: вычисление 2a - x

Если x ≥ a: результат = 8

Вывод результата
\end{verbatim}

\begin{figure}
\centering
\includegraphics{./11.png}
\caption{Результат lab7}
\end{figure}

Выводы

В ходе лабораторной работы были освоены: Команды безусловного перехода
(jmp) Команды условного перехода (jg, jl, je, jz) Работа с файлами
листинга NASM Написание программ с ветвлениями Преобразование данных
между символьным и числовым форматами Анализ машинного кода и адресации
в листинге Работа с языком ассемблера NASM позволяет лучше понять
архитектуру компьютера и принципы работы процессора на низком уровне.
Особенно важным было изучение циклов обработки строк и работы с флагами
процессора.
